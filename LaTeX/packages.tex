% Pacchetti per la formattazione della pagina e del testo
\usepackage{geometry} % Personalizzazione dei margini
\usepackage[small]{titlesec} % Formattazione dei titoli delle sezioni

% Pacchetti per la matematica e la fisica
\usepackage{amsthm} % Teoremi e definizioni matematiche
\usepackage{amsfonts, amssymb, amsmath} % Simboli matematici e font speciali
\usepackage{physics} % Notazione vettoriale e operatori matematici
\usepackage{mathcomp} % Simbolo per mille
\usepackage{mathtools} % Miglioramenti e estensioni a LaTeX per la matematica
\usepackage{cancel} % Cancellazione di termini in formule matematiche

% Pacchetti per la gestione delle figure e delle tabelle
\usepackage{graphicx} % Inclusione di immagini
\usepackage{subfigure} % Creazione di sottofigure

% Pacchetti per la gestione degli elenchi e delle enumerazioni
\usepackage{enumitem} % Personalizzazione degli elenchi
\usepackage{multirow} % Unione di righe nelle tabelle

% Pacchetti per la gestione delle note a piè di pagina e dei riferimenti
\usepackage[toc, page]{appendix} % Gestione degli allegati
\usepackage[square, sort, comma, numbers]{natbib} % Stile di citazione bibliografica
\usepackage{hyperref} % Gestione dei collegamenti ipertestuali e degli URL
\usepackage{epigraph} % Citazioni iniziali dei capitoli
\usepackage{fancyhdr} % Personalizzazione degli stili di pagina
\usepackage{sidecap} % Didascalie laterali delle figure

% Pacchetti per la gestione dei colori e delle decorazioni
\usepackage{xcolor} % Gestione dei colori

% Pacchetti per la gestione dei file di input e degli indici
\usepackage[utf8]{inputenc} % Codifica dei caratteri
\usepackage{makeidx} % Creazione di indici

% Pacchetti per la gestione delle modifiche ai font
\usepackage{bbold} % Font matematico
\usepackage[T1]{fontenc} % Codifica dei font

\usepackage{fancyhdr}