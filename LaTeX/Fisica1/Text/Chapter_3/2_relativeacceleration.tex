\section{Teorema delle accelerazioni relative}
%--------------------------------------------------------------------------
Analogamente al caso delle velocità se definiamo $\vec a_0 = \frac{d\vec u}{dt}$
ed $\vec a' = \ddot x'\hat u_{x'}+\ddot y'\hat u_{y'}+\ddot z'\hat u_{z'}$
possiamo scrivere che:

\begin{equation}
    \frac{d\vec v'}{dt} = \vec a' + \vec\omega\times\vec v' = \vec a - \vec a_0
    -\frac d{dt}\sx\vec\omega\times\vec r'\dx
\end{equation}
\\
\begin{equation}
    \frac d{dt}\sx\vec\omega\times\vec r'\dx = \frac{d\vec\omega}{dt}\times\vec r'
    + \vec\omega\times\frac{d\vec r'}{dt} = \vec\alpha\times\vec r' + \vec\omega\times
    \sx\vec v' + \vec\omega\times\vec r'\dx
\end{equation}
\\
\begin{equation}
    \boxed{\vec a' = \vec a - \vec a_0 - 2\vec\omega\times\vec v'-\vec\alpha\times
    \vec r' - \vec\omega\times\vec\omega\times\vec r'}
\end{equation}
\\
In questo caso l'accelerazione di trascinamento è definita come $\vec a_t = 
\vec a_0 + \vec\alpha\times\vec r' +\vec\omega\times\vec\omega\times\vec r'$,
quindi in definitiva avremo che:

\begin{equation}
    \vec a' = \vec a - \vec a_t -2\vec\omega\times\vec v'
\end{equation}
\\
Abbiamo quindi disaccoppiato da $\vec a_t$ un termine dipendente dalla velocità
del punto in $\mathcal{S'}$, chiamato accelerazione di Coriolis.

\begin{equation}
    \vec a_{cor} = 2\vec\omega\times\vec v'\seg \boxed{\vec a' = \vec a -\vec a_t
    -\vec a_{cor}}
\end{equation}