\section{Sistemi inerziali e non inerziali}
%---------------------------------------------------------------------------
Possiamo classificare i sistemi di riferimento in due macrogruppi: inerziali
e non inerziali.\\
Un sistema è detto inerziale se in assenza di forze esterne applicate un corpo
si muove di moto rettilineo uniforme. Ovvero vale il primo principio della
dinamica e di conseguenza un sistema è inerziale rispetto ad un altro se si
muove di moto rettilineo uniforme rispetto a quest'ultimo.\\
Tutti i sistemi che non rispettano queste condizioni sono detti non inerziali.

\subsection{Inerziali}

Se $\vec r_{0(t)} = \vec r_0 + \vec u t$ avremo che $\vec a_0 = \vec0$ ed
anche $\vec\omega = \vec 0$, quindi la velocità e l'accelerazione misurate
in $\mathcal{S'}$ saranno:

\begin{equation}
    \vec v' = \vec v - \vec u \seg \vec a' = \vec a
\end{equation}
\\
Tra sistemi inerziali dunque, le velocità di sottraggono mentre le accelerazioni
sono le stesse, ne segue che anche le forze sono le stesse.

\begin{equation}
    \vec F = m\vec a\qquad \vec F' = m\vec a' = m\vec a = \vec F
\end{equation}
\\
Concludiamo quindi che i sistemi di riferimento inerziali sono equivalenti dal
punto di vista meccanico, e non esistono dunque esperimenti che possono indicare
quale de due sistemi sia in movimento.

\subsection{Non inerziali}

Nei sistemi non ineziali troveremo $\vec a_t \ne \vec0$ e/o $\vec\omega \ne \vec 0$,
quindi $\vec a' = \vec a$.

\begin{equation}
        \vec F' = m\vec a' = m\vec a -m\vec a_t - m\vec a_{cor} = \vec F -
        \vec F_t -\vec F_{cor}
\end{equation}
\\
Come possiamo notare anche in assenza di forze esterne applicate, con
$\vec F = \vec 0$, si avrà $\vec F' = - \vec F_t -\vec F_{cor}$. Compaiono
quindi le forze inerziali che alterano apparentemente il moto rettilineo
uniforme degli oggetti.