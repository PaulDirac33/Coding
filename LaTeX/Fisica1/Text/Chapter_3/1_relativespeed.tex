\section{Teorema delle velocità relative}

Dati due differenti sistemi di riferimento,  $\mathcal{S}$ in quiete, ed
$\mathcal{S'}$ in moto rispetto ad  $\mathcal{S}$, si vogliono trovare le
relazioni che legano le grandezze osservate nel sistema fisso e quelle
osservate nel sistema mobile.\\
Quindi preso un punto generico $P$ nello spazio, esso avrà come raggio
vettore $\vec r$ in  $\mathcal{S}$, ed $\vec r'$ in  $\mathcal{S'}$.
\\Possiamo dunque scrivere che:

\begin{equation}
    \vec r - \vec{OO'} = \vec r' \seg \boxed{ \vec r' = \vec r - \vec r_{0(t)}}
\label{eq:r_prime}
\end{equation}
\\
Dove $\vec{OO'}$, chiamato d'ora in poi $\vec r_{0(t)}$, è il raggio vettore
dell'origine del sistema $\mathcal{S'}$.
Derivando rispetto al tempo l'equazione (\ref{eq:r_prime}) si otterrà la
velocità del punto $P$ nel sistema $\mathcal{S'}$ in funzione della velocità
in $\mathcal{S}$, della velocità di traslazione del sistema $\mathcal{S'}$
ed in generale anche della velocità angolare con cui ruota $\mathcal{S'}$.

\begin{equation}
    \frac{d\vec r'}{dt} = \frac{d\vec r}{dt} - \frac{d\vec r_0}{dt}
\end{equation}
\\
Definiamo con $\vec u = \frac{d\vec r_0}{dt}$ la veloctià dell'origine $O'$,
mentre dato che $\vec r' = x'\hat u_{x'} + y'\hat u_{y'} + z'\hat u_{z'}$,
occorre derivare anche i versori direzionali del sistema $\mathcal{S'}$.

\begin{equation}
    \frac{d\vec r'}{dt} = \dot x'\hat u_{x'}+\dot y'\hat u_{y'}+\dot z'\hat u_{z'}+
    + x'\frac{d\hat u_{x'}}{dt}+ y'\frac{d\hat u_{y'}}{dt}+ z'\frac{d\hat u_{z'}}{dt}
    =\vec v-\vec u
\end{equation}
\\
Quindi definendo $\vec v' = \dot x'\hat u_{x'}+\dot y'\hat u_{y'}+
\dot z'\hat u_{z'}$ e ricordando che $\frac{d\hat u_{x'_i}}{dt} =
\vec\omega\times\hat u_{x'_i}$ otterremo il teorema delle velocità relative:

\begin{equation}
    \boxed{\vec v' = \vec v - \vec u - \vec \omega\times\vec r'}
\end{equation}
\\
\begin{equation}
    \vec v_t \coloneqq \vec v - \vec v' = \vec u + \vec \omega\times\vec r'
    \seg \boxed{\vec v' = \vec v - \vec v_t}
\end{equation}