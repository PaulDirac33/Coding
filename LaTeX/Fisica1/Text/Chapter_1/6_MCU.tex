\section{Moto circolare uniforme}
%-----------------------------------------------------------------------------
Un moto circolare è un moto planare in cui il punto materiale descrive come
traiettoria una circonferenza. Esso può essere scomposto in due moti armonici. Il moto circolare uniforme è chiamato in questo modo perché avviene a velocità angolare $(\omega)$ costante.
Usiamo come coordinate principali quelle curvilinee.

\begin{equation}
    s_{(t)} = R\phi_{(t)}\seg \phi = \frac sR\seg \omega = \dot \phi = \frac vR
\end{equation}
\\
Otteniamo la relazione:
\begin{equation}
    \boxed{v = \omega R}
\label{eq:v=wr}
\end{equation}
\\
Ora esattamente come abbiamo fatto nel paragrafo $(1.1)$ calcoliamo la 
legge oraria per la variabile $\phi$.

\begin{equation}
    \omega_{(t)} = \dot\phi \seg \phi_{(t)} = \phi_0 + \int_{t_0}^{t}dt'
    \omega_{(t')}\quad\quad \omega_{(t)} = \omega_0
\end{equation}
\\
\begin{equation}
    \boxed{\phi_{(t)} = \phi_0 +\omega_0\sx t - t_0\dx}
\label{eq:MCU}
\end{equation}
\\
\begin{equation}
    v = \omega_0R\seg a = a_n = \frac{v^2}R = \omega_0^2R
\label{eq:an}
\end{equation}
\\
Il periodo di rotazione è:
\begin{equation}
    T = \frac1\nu = \frac{2\pi}{\omega_0} = \frac{2\pi R}v
\label{eq:period}
\end{equation}