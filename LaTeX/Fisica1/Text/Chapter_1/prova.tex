\chapter{Cinematica del punto materiale}
%-----------------------------------------------------------------------------------------------------------------------------------

\begin{figure}[htbp]
\begin{center}
\includegraphics[width=10cm]{images/assi.png} 
\caption{Sistema di riferimento cartesiano: punto materiale $(P)$, raggio vettore $\vec r$.}
\label{default}
\end{center}
\end{figure}



\section{Moto rettilineo uniforme}
Un moto rettilineo uniforme è un moto che avviene lungo una retta a velocità costante, dunque può sempre essere ricondotto ad un moto unidimensionale.\\
Supponiamo di avere dunque un punto materiale nello spazio, che si muove di velocità generica costante. Ricaviamo le equazioni del moto: