\section{Moto rettilineo uniforme}
%---------------------------------------------------------------------------
Un moto rettilineo uniforme è un moto che avviene lungo una retta
a velocità costante,
dunque può sempre essere ricondotto ad un moto unidimensionale.\\
Supponiamo di avere dunque un punto materiale nello spazio,
che si muove di velocità generica costante $v_0$.\\
Ricaviamo le equazioni del moto:

\begin{equation}
    \vec v_{(t)} = \vec v_0 = cost\quad\quad\quad
    \vec v_{(t)} = \frac{d\vec x}{dt}
\label{eq:velocity}
\end{equation}
\\
Integrando tra l'istante iniziale e quello finale otterremo la
legge oraria di un moto rettilineo uniforme:

\begin{equation}
    \vec x_{(t)} = \vec x_0 + \int_{t_0}^{t}dt' \vec v_{(t')}\seg
    \boxed{\vec x_{(t)} = \vec x_0 +\vec v_0\sx t - t_0\dx}
\label{eq:MRU}
\end{equation}
\\
Dove $t_0$ ed $\vec x_0$ sono rispettivamente, istante e
posizione iniziali.