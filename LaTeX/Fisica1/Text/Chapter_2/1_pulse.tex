\section{Quantità di moto}
%---------------------------------------------------------------------------
Una grandezza molto importante che si introduce in dinamica è la quantità
di moto o impulso o momento (lineare). La quantità di moto di un corpo è
definita come il prodotto tra la sua massa e la sua velocità, dunque nel
sistema internazionale si misura in chilogrammi per metro al secondo.
\begin{equation}
    \boxed{\vec p = m\vec v}\quad\quad \sss p\ddd = M\cdot L\cdot T^{-1}
\label{eq:pulse_def}
\end{equation}
Nel caso in cui la massa dell'oggetto sia costante nel tempo, ovvero nella
maggior parte dei casi che incontreremo, derivando l'impulso rispetto al
tempo otterremo che:
\begin{equation}
    \frac{d\vec p}{dt} = \frac d{dt}\sx m\vec v\dx =
    m\frac{d\vec v}{dt} =\vec F
\end{equation}
Quindi la seconda legge di Newton nel caso più generale, identifica la
forza come la derivata temporale della quantità di moto.
\begin{equation}
    \boxed{\vec F = \frac{d\vec p}{dt}}
\label{eq:newt2law_p}
\end{equation}
\subsection{Teorema dell'impulso}
Integrando l'equazione (\ref{eq:newt2law_p}) possiamo ricavare la variazione della
quantità di moto tra istante finale e istante iniziale che chiameremo impulso $(\vec J)$.
\begin{equation}
    \vec F = \frac{d\vec p}{dt}\seg \int_{t_0}^tdt'\frac{d\vec p}{dt'} =
    \int_{t_0}^tdt'\vec F\seg \vec p_{(t)} - \vec p_{\sx t_0\dx} =
    \int_{t_0}^tdt'\vec F
\end{equation}
\begin{equation}
    \boxed{\vec J = \int_{t_0}^tdt'\vec F_{(t')}}
\label{eq:pulse_theorem}
\end{equation}
\\
Per il teorema della media, il valor medio della forza  è proprio il rapporto tra variazione di quantità di moto e intervallo di tempo.
\begin{equation}
\vec F_m = \frac1{t-t_0}\int_{t_0}^tdt'\vec F = \frac{\Delta\vec p}{\Delta t}
\end{equation}
\begin{itemize}
    \item Se $\vec F$ è costante nel tempo allora:
    \begin{equation}
        \vec J = \vec F\Delta t\seg\Delta\vec p = \vec J \seg\Delta\vec v
        = \frac{\vec J}m
    \end{equation}
    \\\item Se $\vec F$ è nulla allora:
    \begin{equation}
        \Delta\vec p = 0\seg\vec p_{(t)} = cost
    \end{equation}
    In assenza di forze applicate la quantità di moto di un corpo si conserva.
\end{itemize}